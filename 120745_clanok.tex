\documentclass[10pt,twoside,english,a4paper]{article}

\usepackage[english]{babel}
\usepackage{fontenc} 
\usepackage{inputenc}
\usepackage{graphicx}
\usepackage{url} 
\usepackage{hyperref} 
\usepackage{cite}
\usepackage{titlesec}

\pagestyle{headings}
\title{Role-playing games, their development and interactive storytelling
\thanks{Semester project in the subject Methods of engineering work, if. year 2022/23, management: Vladimír Mlynarovič }}

\author{Gabriel Ábrahám\\[2pt]
	{\small Slovenská technická univerzita v Bratislave}\\
	{\small Fakulta informatiky a informačných technológií}\\
	{\small \texttt{xabraham@stuba.sk}}	
	}

\date{\small 5. november 2022}


\begin{document}

\maketitle

\begin{abstract} %fix abstract
The development of tabletop role-playing games to modern computer role-playing games was enormous, introducing a new and different experience from books for getting immersed in a story. This work introduces RPGs, and their genres in-depth and presents their elements in some examples, elaborating on their history/development from physical tabletop roleplaying to software-based RPGs with many genres including the adaptation which is the closest to the tabletop version. Also introduces their interactive storytelling, which is one of their most famous elements, and why it could also serve as an alternative for classic reading.

\end{abstract}

\pagebreak

\section*{Introduction}
The widespread game genre, which goes by the name role-playing games has been present as an idea for a relatively long time. ~\ref{def} It all started around the 1970s, when tabletop rpgs (also known as pen-and-paper rpgs) started appearing with Dungeons \& Dragons as the most known one  ~\ref{tabletop}.\\
After advancements in information technology these games started appearing in a modern version, in the form of video games with many subgenres ~\ref{genres}, retaining lots of elements from their predecessor, but  also introducing new ones thanks to technology ~\ref{elements}.\\

\section{Definition} \label{def}
 Defining what counts as an RPG game has become a very difficult task as the genre itself is quite large with a lot of game elements that are nowadays adapted in games of other genres. It's generally described as \cite{CRPG} a game in which layers advance through a story quest, and often many side quests, for which their character or party of characters gain experience that improves various attributes and abilities. \cite{RPGs} For the game to work as an aesthetic experience players must be willing to bracket their natural selves and enact a fantasy self. They must lose themselves to the game which can be quite easy if the world and the story of the game is good enough.


\section{Tabletop role-playing games} \label{tabletop}

The predecessor of modern role-playing games were tabletop RPGs which are still popular nowadays among many people. The general idea of the game is that you have a game master who is basically the storyteller and the one who is pulling the strings behind the walls. There are many different kinds of dice used during the game, like the most popular one, the D20, which has 20 sides or the D3 with 3 sides.  The outcome of actions, e.g. throwing a fireball, hitting an enemy is based on dice rolls simulating randomness, which make the game unpredictable and fun. Usually the more powerful an action is, the more risk is to it, possibly affecting you or your party in a bad way, which might have a huge effect on the outcome of your adventure. Of course there is not only fighting, but general conversations too where you can actively change the story too as you'd like. This is why tabletop RPGs were always a great way to spend time.

\section{Genres} \label{genres}

Like I have already mentioned RPGs have a lot of subgenres, which have appeared during the ages, such as:
\begin{itemize}
  \item CRPG - Classical Role-playing games - The genre, which stands the closest to pen and paper RPGs using most of their elements for which we can call them their spiritual successors too. Most popular titles would be the Baldur's Gate and Divinity~\ref{elements} game series , which will be mentioned in greater detail later. 
  \item JRPG - Japanese Role-playing games - Generally have a turn-based system of combat and a linear story and gameplay with a pre-determined player character (e.g. Pokemon series, Octopath Traveller)
  \item ARPG - Action Role-playing games - Put a lot of emphasis on fast paced real time combat (e.g. Diablo series)
  \item MMORPG - Massively multiplayer online role-playing games - Usually high importance of exploration with real time combat in a multiplayer open world (e.g. World of Warcraft)
\end{itemize} 

\section{Elements} \label{elements}

Moving on to elements of role-playing games, the main focues is going to be on Divinity: Original Sin II and it's gameplay elements and how they affect the story.


  \subsection{Attributes}

The game offers six different main attributes which we can raise by spending attribute points, which we get after every level we pass. These points grant bonuses to our characters as follows:

\begin{itemize} %insert statistic about most popular choice among general people 		%insert pictures
  \item Strength - increases the weight our character can carry and allows us to move heavier objects, as well as increasing the damage weapons such as spear and swords deal
  \item Finesse - increases the damage weapons such as bows and daggers deal
  \item Intelligence - increases the damage weapons such as wands and staves deal and in addition increases the power of spells
  \item Constitution - determines how much Vitality our character has. If characters run out of Vitality, they die
  \item Memory - determines how many skills can be equipped for use in combat.
  \item Wits - raises the chance of scoring critical hits as well as raising the ability to detect traps and hidden treasures 
\end{itemize} 
In addition to these six we can also raise our persuasion(and many more civic abilities), which comes in handy during conversations. With it's help we can most of the time turn the tide in our favor. If you for example want to intimidate a someone you'd need some persuasion skills and in addition to that strength too. If you try to intimidate someone without having the sufficient skills for it, you will fail and it is going to have it's consequences. This is where we are beginning to see the interactive storytelling of the game. It's all on you to decide what you want to do in every situation and by adding another level of complexity by the main attributes not only increasing your damage and vitality, but also dictating how characters act and what they can do outside of combat, we get a complex game system, which can affect our whole story based on all of our decisions.
   \subsection{Friendship}
In addition to attributes, the game introduces a friendship level too. If it's too low with a non-player character the character is going to consider our character an enemy, thus attacking us. The friendship level is decreased by failed attempts of stealing things or making decisions that upset the character in question. On the other hand it can be increased by fulfilling the characters wishes, completing their quests or simply giving them money.

\bibliography{literatura}
\bibliographystyle{abbrv}

\end{document}
